%%%%%%%%%%%%%%%%%%%%%%%%%%%%%%%%%%%%%%%%%%%%%%%%%%%%%%%%%%%%

\section{Conclusion}
\label{S:concl}

\vspace{-27\in}

Any domain with large data to explore (`large' in terms of what the user needs to understand) may benefit from interactive exploration possibly with editing or annotation; see tools for ontologies~\cite{BaskayaKJ10}, graphs~\cite{HaunNKTB10}, semantic data~\cite{DumasBHS12}, software bugs~\cite{HoraADBCVM12}, API usage~\cite{RooverLP13}. In this paper, we described an approach to the exploration of \Wikipedia's category graph so that candidate taxonomies can be extracted from the graph. We were specifically interested in understanding \Wikipedia's classification of software languages. To this end, we developed a domain-specific exploration tool, \WikiTax, which supports level-by-level graph extraction, metrics-based graph visualization as well as transparent and revisable graph reduction. Such designated exploration support is missing in more generic tools for searching or exploring the category graph.

The described method of graph reduction is deliberately interactive and relies on domain knowledge for transparent exclusion decisions, as opposed to any means of automated ontology extraction / generation~\cite{SuchanekKW08,WuW08}. (Without such validation, there is little hope that the resulting taxonomy would be readily meaningful.) An important conceptual contribution is our proposal to document exclusion decisions with (comments for) exclusion types, thereby making reduction more systematic and transparent. This interactive approach can be contrasted with related work on taxonomy or ontology mining, where categories are classified and additional relationships are inferred automatically, e.g., by analyzing the structure of compound category names~\cite{NastaseS08}.

We contend that the described approach provides the initial core of a method for actually developing a taxonomy for software languages (and possibly other taxonomies) on the grounds of \Wikipedia. Collaborative work and further improved tool support are needed to actually arrive at a comprehensive taxonomy. We imagine that we need powerful refactoring operations on the category graph to facilitate taxonomy extraction and enforcement of consistent style. The exploration of the category graph could also be supported by additional forms of visualization, e.g., for understanding the overlap of categories. Also, we need to generally better understand (perhaps based on an automated analysis) the different classifier styles used by \Wikipedia.

%%%%%%%%%%%%%%%%%%%%%%%%%%%%%%%%%%%%%%%%%%%%%%%%%%%%%%%%%%%%

\begin{comment}
% Another problem with Wikipedia's use of categories
\WikipediaCategory{Ada programming language} is a subcategory of, for example, \WikipediaCategory{Concurrent programming languages}, while \WikipediaPage{Ada (programming language)} is not a member page of 
\WikipediaCategory{Concurrent programming languages}, but it is a member of various other ``... programming languages'' categories. For comparison consider \WikipediaCategory{Erlang programming language}, which is also a subcategory of, for example, \WikipediaCategory{Concurrent programming languages} and \WikipediaPage{Erlang (programming language)} is also a member page of \WikipediaCategory{Concurrent programming languages}.
\end{comment}

%%%%%%%%%%%%%%%%%%%%%%%%%%%%%%%%%%%%%%%%%%%%%%%%%%%%%%%%%%%%
