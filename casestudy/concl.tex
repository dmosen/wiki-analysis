%%%%%%%%%%%%%%%%%%%%%%%%%%%%%%%%%%%%%%%%%%%%%%%%%%%%%%%%%%%%

\section{Conclusion}
\label{S:concl}

We contend that \WikiTax{} is quite helpful in exploring \Wikipedia's category graph and reducing subgraphs to candidate taxonomies. Thus, \WikiTax{} is a highly domain-specific exploration tool. In principle, such exploration could also be performed by means of search engines on Wikipedia (e.g., \cite{MilneW11a}) or plainly programmatically (by writing API-based queries against Wikipedia or DBpedia\footnote{\url{http://dbpedia.org}} or possibly Wikidata\footnote{\url{https://www.wikidata.org/}}), but this path, which we experimented with before designing \WikiTax, would not enable convenient exploration and transparent judgements.

Any domain with large data to explore (`large' in terms what the user needs to understand) requires such interactive exploration tools including features for editing or annotation, see, e.g., tools for ontologies~\cite{BaskayaKJ10}, graphs~\cite{HaunNKTB10}, semantic data~\cite{DumasBHS12}, software bugs~\cite{HoraADBCVM12}, API usage~\cite{RooverLP13}.

The key features of \WikiTax{} are scalability in terms of data access to \Wikipedia{} category graph, navigation thereupon, metrics-based visualization, link support to \Wikipedia, and annotation support for excluding edges in a systematically manner. The proposed paradigm of taxonomy building is deliberately interactive and relies on (transparent) judgements by the user, as opposed to any means of automated ontology extraction / generation~\cite{WuW08,SuchanekKW08}. An important conceptual contribution is our proposal for classifying classifiers, thereby supporting the systematic (transparent) reduction of the category graph. This is again a more judgmental than automatic approach, when compared to related work on taxonomy or ontology mining, where categories are also classified and additional relationships are inferred, e.g., by analyzing the structure of compound category names~\cite{NastaseS08}.

To summarize, we have initiated a path towards derivation of SL(E) taxonomy, thoroughly informed by \Wikipedia. Collaborative work and presumably further tool extensions are needed to actually arrive at a comprehensive taxonomy. We imagine that we need powerful, pattern-based refactoring operations on the category graph to actually obtain a satisfactory taxonomy.

%%%%%%%%%%%%%%%%%%%%%%%%%%%%%%%%%%%%%%%%%%%%%%%%%%%%%%%%%%%%

\begin{comment}
% Another problem with Wikipedia's use of categories
\WikipediaCategory{Ada programming language} is a subcategory of, for example, \WikipediaCategory{Concurrent programming languages}, while \WikipediaPage{Ada (programming language)} is not a member page of 
\WikipediaCategory{Concurrent programming languages}, but it is a member of various other ``... programming languages'' categories. For comparison consider \WikipediaCategory{Erlang programming language}, which is also a subcategory of, for example, \WikipediaCategory{Concurrent programming languages} and \WikipediaPage{Erlang (programming language)} is also a member page of \WikipediaCategory{Concurrent programming languages}.
\end{comment}

%%%%%%%%%%%%%%%%%%%%%%%%%%%%%%%%%%%%%%%%%%%%%%%%%%%%%%%%%%%%
