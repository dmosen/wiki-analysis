\documentclass[a4paper,runningheads]{myllncs}

\usepackage[utf8]{inputenc}
\usepackage[T1]{fontenc}
\usepackage{lmodern}
\usepackage[english]{babel}
\usepackage{hyperref}
\usepackage{comment}
\usepackage{boxedminipage}
\usepackage{graphicx}
\usepackage{booktabs}
\usepackage{wrapfig}
\usepackage{listings}
\usepackage{xcolor}
\usepackage{rotating}
\usepackage{xstring}

\newcommand{\ooo}[1]{\textsf{101#1}}
\newcommand{\oneohone}{\ooo{}}
\newcommand{\MediaWiki}{\textsf{MediaWiki}}
\newcommand{\Wikipedia}{\textsf{Wikipedia}}
\newcommand{\WikipediaUrl}{http://en.wikipedia.org/wiki}
\newcommand{\WikipediaPage}[1]{%
  \protect\StrSubstitute{#1}{ }{_}[\wikiurl]%
  \protect\href{\WikipediaUrl/\wikiurl}{\emph{#1}}}
\newcommand{\WikipediaCategory}[1]{%
  \protect\StrSubstitute{Category:#1}{ }{_}[\wikiurl]%
  \protect\href{\WikipediaUrl/\wikiurl}{\emph{#1}}}
\newcommand{\WikiTaxEmptyCell}{--}
\newcommand{\WikiTax}{\textsf{WikiTax}}
\newcommand{\WikiTaxCategory}[1]{\WikipediaCategory{#1}}
\newcommand{\WikiTaxSubcategory}[1]{\WikipediaCategory{#1}}
\newcommand{\WikiTaxComment}[1]{#1}
\newcommand{\WikiTaxNewLine}{\\\hline}

%\clubpenalty = 10000 
%\widowpenalty = 10000 
%\displaywidowpenalty = 10000

\colorlet{punct}{red!60!black}
\definecolor{background}{HTML}{EEEEEE}
\definecolor{delim}{RGB}{20,105,176}
\colorlet{numb}{magenta!60!black}

\lstdefinelanguage{json}{
    basicstyle=\normalfont\ttfamily\scriptsize,
    showstringspaces=false,
    breaklines=true,
    frame=lines,
    backgroundcolor=\color{background},
    literate=
     *{0}{{{\color{numb}0}}}{1}
      {1}{{{\color{numb}1}}}{1}
      {2}{{{\color{numb}2}}}{1}
      {3}{{{\color{numb}3}}}{1}
      {4}{{{\color{numb}4}}}{1}
      {5}{{{\color{numb}5}}}{1}
      {6}{{{\color{numb}6}}}{1}
      {7}{{{\color{numb}7}}}{1}
      {8}{{{\color{numb}8}}}{1}
      {9}{{{\color{numb}9}}}{1}
      {:}{{{\color{punct}{:}}}}{1}
      {,}{{{\color{punct}{,}}}}{1}
      {\{}{{{\color{delim}{\{}}}}{1}
      {\}}{{{\color{delim}{\}}}}}{1}
      {[}{{{\color{delim}{[}}}}{1}
      {]}{{{\color{delim}{]}}}}{1},
}

\author{Ralf L\"ammel \and Dominik Mosen and Andrei Varanovich}
\institute{University of Koblenz-Landau, Software Languages Team}
\title{Wikipedia's software language taxonomy}
\subtitle{Tool demonstration}

\begin{document}

\maketitle

\begin{abstract} 

  \Wikipedia{} provides useful input for efforts on mining taxonomies or ontologies. In particular, the category graph can be viewed as hinting at a description of a taxonomy. In this paper, we describe a workflow and corresponding tool support for exploring \Wikipedia's category graph so that a candidate taxonomy for software languages can be derived and evaluated. The \WikiTax{} tool supports exploration of \Wikipedia's category graph in an interactive manner such that it is rendered in a tree-like manner, irrelevant nodes and edges may be removed, comments on judgmental decisions may be added, and the result is visualized based on graph-based metrics. The tool demonstration focuses on the actual application of the tool to \Wikipedia's categories for computer and software languages.

\end{abstract}

%%%%%%%%%%%%%%%%%%%%%%%%%%%%%%%%%%%%%%%%%%%%%%%%%%%%%%%%%%%%

%%%%%%%%%%%%%%%%%%%%%%%%%%%%%%%%%%%%%%%%%%%%%%%%%%%%%%%%%%%%

\section{Introduction}
\label{S:intro}

Ever since 2008, the calls for papers for the \emph{Software Language Engineering} (SLE) conference\footnote{\url{http://planet-sl.org/}} have contained slightly different, more implicit or more explicit definitions of the term `software language'. Other community material contains yet other definition attempts; see, for example, the IEEE TSE special section on SLE in 2009~\cite{FavreGLW09}. At SLEBOK 2012 (i.e., an SLE 2012 satellite event dedicated to the the SL(E) body of knowledge), the attendees were also getting into the issue of what exactly a software language is and what classification may help in arriving at an accepted comprehensive definition.

Some classes of software languages are generally agreed upon. For instance, programming languages are definitely software languages; they are conceptually well understood and characterized in terms of programming language concepts; see, for example, textbooks on programming languages, programming paradigms, and programming language theory, e.g., \cite{Mosses92,Sebesta12}. Some classes of languages have been the target of scholarly work on language classification; see, for example, classifications of model transformation languages~\cite{CzarneckiH06}, business rule modeling languages~\cite{SkalnaG12}, visual languages~\cite{BottoniG04,MarriottM97,BurnettB94}, architecture description languages~\cite{MedvidovicT00}, and programming languages~\cite{BabenkoRY75,DoyleS87}.

In our work on the software chrestomathy `\oneohone'~\cite{FavreLV12}\footnote{\url{http://101companies.org/}}, we also aim at the classification of software languages, but we have failed to make a serious proposal so far. We are simply not confident regarding classification style, expected level of detail, and treatment of multiple dimensions of classification. In fact, such a SL(E) classification challenge is by no means limited to software languages; it also applies to \emph{software technologies} and \emph{software concepts}. Perhaps, we may need to lower expectations and accept the use of simpler tagging schemes (as used on StackOverflow, for example) as opposed to hierarchically organized, consistent and comprehensive taxonomies.

\Wikipedia{} contains substantial amounts of taxonomy-like (if not ontology-like) information---also for software languages, technologies, and concepts. For instance, there are hierarchically organized categories such as \WikipediaCategory{Computer languages}, \WikipediaCategory{Programming languages}, and \WikipediaCategory{Programming language classification}, but yet other roots for exploration may be reasonable. We suggest that the SL(E) classification challenge shall be informed by the exploration of \Wikipedia. 

In this paper, we describe support for such exploration based on the \WikiTax{} tool that was developed exactly for this use case. We also demonstrate exploration, without though any claim of having found a good candidate taxonomy for SL(E). Rather the expectation is that \WikiTax{} and the associated paradigm add more structure to the ongoing community effort of addressing the SL(E) classification challenge. The source code of \WikiTax, a comprehensive manual, and all data covered in this paper are available online.\footnote{\url{https://github.com/dmosen/wiki-analysis}}

\paragraph*{Road-map} \S\ref{S:tool} describes the \WikiTax{} tool. \S\ref{S:study} explores some \Wikipedia{} categories related to software languages. \S\ref{S:concl} concludes the paper.

%%%%%%%%%%%%%%%%%%%%%%%%%%%%%%%%%%%%%%%%%%%%%%%%%%%%%%%%%%%%

%%%%%%%%%%%%%%%%%%%%%%%%%%%%%%%%%%%%%%%%%%%%%%%%%%%%%%%%%%%%

\section{Exploring \Wikipedia{} with \WikiTax} 
\label{S:tool}

\paragraph*{\textbf{\Wikipedia's category graph}}

\Wikipedia{} uses several means of organizing its information: plain links giving rise to an article graph, designated article lists, portals meant to introduce users to key topics, infoboxes for semantic (`typed') data, and categories giving rise to a category graph for the classification of articles. When it comes to taxonomy mining, the category graph is particularly relevant; the graph is accessible, for example, through the \MediaWiki{} API\footnote{\url{http://www.mediawiki.org/wiki/API:Main_page}}, which is the access path chosen by \WikiTax.

%%%%%%%%%%%%%%%%%%%%%%%%%%%%%%%%%%%%%%%%%%%%%%%%%%%%%%%%%%%%

\paragraph*{\textbf{Graph extraction and reduction with \WikiTax}}

Initially, \WikiTax{} is pointed to a root category (level 0) for extraction. Iteratively, subcategories and pages (in fact, page titles) can be extracted level by level or exhaustively. Exhaustive extraction may take minutes our hours depending the root category. The \Wikipedia{} category graph contains many surprising edges, which easily implies inclusion of large irrelevant subgraphs. 

\WikiTax{} supports reduction of the graph both along level-by-level extraction and post extraction. Reduction involves the selection of edges exclusion. If all edges to a given category are excluded, then the corresponding category node also becomes excluded. (We node that a category may have multiple parent categories.) If reduction is applied post extraction, the exclusion is actually implemented as blacklisting. In this manner, all decisions can be easily revisited and adapted.

%%%%%%%%%%%%%%%%%%%%%%%%%%%%%%%%%%%%%%%%%%%%%%%%%%%%%%%%%%%%

\begin{figure}[t!]
\begin{center}
\includegraphics[width=.84\textwidth]{figures/clLevel12.png}
\end{center}
\vspace{-66\in}
\caption{Exploration of level 1 and 2 subcategories of \emph{Computer languages}.}
\label{F:clLevel12}
\vspace{-42\in}
\end{figure}

%%%%%%%%%%%%%%%%%%%%%%%%%%%%%%%%%%%%%%%%%%%%%%%%%%%%%%%%%%%%

\autoref{F:clLevel12} shows the \WikiTax{} exploration view in the following state.
Two levels (level 1 and 2) were extracted starting from the category \Wikipedia{Computer languages}. (That is, we are about to hit the `removal/blacklist' button.) Some edges are already selected for exclusion. In \S\ref{S:study}, we discuss reasons for exclusion systematically, but it suffices here to say that the selected categories are not proper language classifiers. The categories are highlighted according to the metrics of immediate member pages. We have selected the category \WikipediaCategory{Articles with example code} for which some extra data is shown in the panel on the right. All categories and pages are clickable to navigate to \Wikipedia.

%%%%%%%%%%%%%%%%%%%%%%%%%%%%%%%%%%%%%%%%%%%%%%%%%%%%%%%%%%%%

\begin{figure}[ht]
\centering
\includegraphics[width=0.63\textwidth]{../manual/figures/full_schema.pdf} 
\caption{Metamodel of the \WikiTax{} category graph.}
\label{F:metamodel}
\vspace{-42\in}
\end{figure}

%%%%%%%%%%%%%%%%%%%%%%%%%%%%%%%%%%%%%%%%%%%%%%%%%%%%%%%%%%%%

\WikiTax{} operates on an enhanced category graph; see the metamodel in \autoref{F:metamodel}. Thus, each category associates with contained pages and subcategories. The subcategory associations are attributed to keep track of metadata as follows: 

\vspace{-22\in}

{\small

\begin{description}
\item[backwardArc] Marker for cyclic edges in the category graph.
\item[blacklisted] Marker for categories blacklisted past extraction.
\item[excluded] Marker for categories excluded during reduction.
\item[comment] Label to be associated with the edge.
\end{description}

}

\vspace{-22\in}

\noindent
Categories are associated with measures as follows:

\vspace{-22\in}

{\small

\begin{description}
\item[level] The level 0, 1, 2, ... of the category in the graph with the root at level 0.
\item[subcategories] The number of immediate subcategories.
\item[transitiveSubcategories] The number of all subcategories.
\item[pages] The number of immediately contained pages.
\item[transitivePages] The number of all pages in this category.
\end{description}

}

\vspace{-22\in}

\noindent
Internally, \WikiTax{} uses the Java-based JGraLab library\footnote{\url{https://github.com/jgralab}} for the representation of (annotated) graphs with JSON as an export format. (Labels are also exposed as CSV.)

%%%%%%%%%%%%%%%%%%%%%%%%%%%%%%%%%%%%%%%%%%%%%%%%%%%%%%%%%%%%

%%%%%%%%%%%%%%%%%%%%%%%%%%%%%%%%%%%%%%%%%%%%%%%%%%%%%%%%%%%%

\section{Explorative study}
\label{S:study}

In this study, we start from the assumed root category for software languages on \Wikipedia, i.e., \WikipediaCategory{Computer languages}. Our objective is to determine a validated category tree of actual classifiers for software languages including measures for the size of the categories.\footnote{All \Wikipedia{} access for this study was validated (again) during 7-18 June 2013 which is also when quotes were extracted from \Wikipedia, as they appear in the text of this section.}

%%%%%%%%%%%%%%%%%%%%%%%%%%%%%%%%%%%%%%%%%%%%%%%%%%%%%%%%%%%%

\subsection{Computer languages: levels 1 and 2}

We begin by pointing \WikiTax{} to \WikipediaCategory{Computer languages}. \WikiTax returns with a small set of immediate subcategories; see \autoref{F:clLevel1}. The figure shows the situation in the tool's dialog past selecting level 1 categories for exclusion in the ultimate category graph. We contend that \WikipediaCategory{Lists of computer languages}, \WikipediaCategory{Articles with example code}, \WikipediaCategory{Data types}, and \WikipediaCategory{Programming language topics} are not concerned with \emph{classification} of languages, and thus, they should be excluded. Arguably, \WikipediaCategory{Data types} could be viewed as computer (or software) languages in the broadest sense, but we did not apply such a broad view.

%%%%%%%%%%%%%%%%%%%%%%%%%%%%%%%%%%%%%%%%%%%%%%%%%%%%%%%%%%%%

\begin{figure}[t!]
\begin{center}
\includegraphics[width=.5\textwidth]{figures/clLevel1.png}
\end{center}
\vspace{-66\in}
\caption{Extraction and reduction of level 1 subcategories for \emph{Computer languages}.}
\label{F:clLevel1}
\vspace{-42\in}
\end{figure}

%%%%%%%%%%%%%%%%%%%%%%%%%%%%%%%%%%%%%%%%%%%%%%%%%%%%%%%%%%%%

\begin{figure}[t!]
{\footnotesize

\begin{center}
\noindent
\begin{tabular}{l|p{3.4in}}
\textbf{Category} & \textbf{Subcategories} \\\hline
\input{../data/Computer_languages.tex-twolevels}
\end{tabular}
\end{center}

\vspace{-42\in}

}
\caption{Reduced subcategory lists for subcategories of \emph{Computer languages}.}
\label{F:twolevels}
\vspace{-42\in}
\end{figure}

%%%%%%%%%%%%%%%%%%%%%%%%%%%%%%%%%%%%%%%%%%%%%%%%%%%%%%%%%%%%

We decided to extract another level to obtain a graph of manageable size. Again, we excluded several categories, if they did not meet our objective of language classification. As a result, we obtained the categories shown in \autoref{F:twolevels}. This is a pretty manageable set of language classifiers. It happens that they all end on ``... languages'' except for two subcategories of \WikipediaCategory{Markup languages} which end on ``... formats''. In contrast, most of the excluded categories (see below) do not end on ``... languages'' .

%%%%%%%%%%%%%%%%%%%%%%%%%%%%%%%%%%%%%%%%%%%%%%%%%%%%%%%%%%%%

\begin{figure}[t!]
{\footnotesize

\begin{center}
\noindent
\begin{tabular}{l|l}
\textbf{Category} & \textbf{Meta classifier} \\\hline
\input{../data/Computer_languages.tex-metaclassify}
\end{tabular}
\end{center}

\vspace{-42\in}

}
\caption{Exclusion summary for levels 1 and 2 of \emph{Computer languages}; this list is produced by the \WikiTax{} tool based on metadata (comments) entered by us interactively.}
\label{F:metaclassify}
\vspace{-42\in}
\end{figure}

%%%%%%%%%%%%%%%%%%%%%%%%%%%%%%%%%%%%%%%%%%%%%%%%%%%%%%%%%%%%

\subsection{Classifier classification}

In order to obtain the reduced result of \autoref{F:twolevels}, we had to exclude 29 categories. This may seem like a small number, but it is clear that we will need to exclude much more categories once we push extraction deeper into the category tree. Thus, we embarked on the classification of reasons for exclusion so that any decision can be labeled accordingly, also suggesting a foundation for reproducing our results. We identified the following classifiers; see \autoref{F:metaclassify} for the full list of excluded categories with the associated classifier:

\medskip

\noindent
\textbf{Alternative classifier}\ \ The category classifies software languages in a manner that is not related to software concepts. For instance, the category \WikipediaCategory{Academic programming languages} describes itself as being concerned with languages that are ``influential in computer science and programming language theory''.

\smallskip

\noindent
\textbf{Deviating classifier}\ \ The category does not actually classify software languages. It rather classifies something else. For instance, category \WikipediaCategory{Articles with example code} describes itself as being concerned with ``articles which include reference implementations of algorithms''.

\smallskip

\noindent
\textbf{Singleton classifier}\ \ The category is effectively concerned with a single software language for which it serves as a container of related entities such as technologies or standards. For instance, category \WikipediaCategory{Cascading Style Sheets} contains pages on all kinds of topics related to the CSS language.

\smallskip

\noindent
\textbf{List classifier}\ \ The category collects lists or categories of lists (rather than plain categories) of software languages. For instance, category \WikipediaCategory{Lists of computer languages} has \WikipediaCategory{Lists of programming languages} as a subcategory, which in turn contains pages for some lists of languages, such as the \WikipediaPage{List of BASIC dialects}.

\smallskip

\noindent
\textbf{Maintenance classifier}\ \ The category is used by the \Wikipedia{} authors to capture some content maintenance-related information. For instance, category \WikipediaCategory{Markup language stubs} describes itself as serving ``for stub articles relating to markup languages''.

%%%%%%%%%%%%%%%%%%%%%%%%%%%%%%%%%%%%%%%%%%%%%%%%%%%%%%%%%%%%

\subsection{Point-wise inquiry}

At this point, exploration already had led to a manageable view on the category graph for software languages. This view is, in fact, quite effective, which we illustrate with an inquiry that suggested itself during exploration.

Looking at \autoref{F:clLevel1} and \autoref{F:twolevels}, we may suspect an asymmetry between `query' versus `transformation'. That is, there is a category \WikipediaCategory{Transformation languages} at level 1, but there is apparently no category for `query languages', not even at level 2. Let us inspect the page for \WikipediaPage{SQL}, which is arguably a quite obvious query language. It turns out that \WikipediaPage{SQL} is a member of various categories including a category \WikipediaCategory{Query languages} which in turn is subcategory of various categories including the category \WikipediaCategory{Domain-specific programming languages} which occurred in \autoref{F:twolevels}. Let us compare this classification scheme with the one of \WikipediaPage{XSLT}, which is arguably a quite obvious transformation language: it is a member of the categories \WikipediaCategory{Transformation languages}, \WikipediaCategory{Declarative programming languages}, \WikipediaCategory{Functional languages}, \WikipediaCategory{Markup languages}, \WikipediaCategory{XML-based programming languages}, and yet other categories that may count as `alternative classifiers'. However, \WikipediaPage{XSLT} (unlike \WikipediaPage{SQL}) is not a member of the category \WikipediaCategory{Domain-specific programming languages}.

We take this sort of observation to mean that the derivation of a highly consistent taxonomy for software languages would require some non-trivial effort in defining and enforcing principles. We were not able to observe this situation so clearly prior to using \WikiTax.

%%%%%%%%%%%%%%%%%%%%%%%%%%%%%%%%%%%%%%%%%%%%%%%%%%%%%%%%%%%%

\subsection{Programming languages: all levels}

\autoref{F:metaclassify} makes it obvious that the subcategory of \WikipediaCategory{Computer languages} with by far the most subcategories is 
\WikipediaCategory{Programming languages}. Thus, we embarked on a more comprehensive exploration of category \WikipediaCategory{Programming languages}:

%%%%%%%%%%%%%%%%%%%%%%%%%%%%%%%%%%%%%%%%%%%%%%%%%%%%%%%%%%%%

\begin{figure}[t!]
\parbox{.48\textwidth}{
Pages

\noindent 
\includegraphics[width=0.41\textwidth]{figures/plPagesTransitive.png}
}\hfill\parbox{.48\textwidth}{

Categories 

\noindent \includegraphics[width=0.41\textwidth]{figures/plSubcategoriesTransitive.png}
}

\vspace{-42\in}

\caption{Metrics-based views on \emph{Programming languages} graph.}
\label{F:plNumbers}
\vspace{-42\in}
\end{figure}

%%%%%%%%%%%%%%%%%%%%%%%%%%%%%%%%%%%%%%%%%%%%%%%%%%%%%%%%%%%%

Initially, we extracted 423 categories over 8 levels with 7515 pages. The automatic extraction took several minutes. We performed exclusion in two steps. First, we excluded direct subcategories of the category \WikipediaCategory{Programming languages}---based on the list in \autoref{F:metaclassify}. After such initial pruning, 288 categories with 6671 pages remained. We completed reduction at all levels of the category graph. This process required about 2 hours of manual work---work that is mainly concerned with checking assumptions for exclusion by consulting corresponding category pages on \Wikipedia. Ultimately, 79 categories over 4 levels with 1560 pages remained.  \autoref{F:plNumbers} visualizes the reduced taxonomy while applying two different metrics, as supported by \WikiTax.

On the left, the metric for the \emph{number of transitive member pages} is applied for visualization. No category is grayed out, which means that there is no category without members. Most of the categories are shown in a plain font, which means that they all carry members, but less than 25\,\% of the total members in the category \WikipediaCategory{Programming languages} (which has 1560 member pages). There is actually one heavyweight: category \WikipediaCategory{Domain-specific programming languages} carries 976 members, which is more than 50\,\% of all members; this status is expressed by 
highlighting the category.

On the right, the metric for the number of transitive subcategories is applied for visualization. Most subcategories of \WikipediaCategory{Programming languages} do not have any subcategories; thus, they are grayed out. 7 out of 36 level-1 categories carry subcategories. 6 out of these 7 categories carry only very few subcategories (less than 5). Category \WikipediaCategory{Domain-specific programming languages} carries 18 subcategories, which is more than 25\,\% of all subcategories; this status is expressed by highlighting the category.

%%%%%%%%%%%%%%%%%%%%%%%%%%%%%%%%%%%%%%%%%%%%%%%%%%%%%%%%%%%%

%%%%%%%%%%%%%%%%%%%%%%%%%%%%%%%%%%%%%%%%%%%%%%%%%%%%%%%%%%%%

\section{Conclusion}
\label{S:concl}

\vspace{-27\in}

Any domain with large data to explore (`large' in terms of what the user needs to understand) may benefit from interactive exploration possibly with editing or annotation; see tools for ontologies~\cite{BaskayaKJ10}, graphs~\cite{HaunNKTB10}, semantic data~\cite{DumasBHS12}, software bugs~\cite{HoraADBCVM12}, API usage~\cite{RooverLP13}. In this paper, we described an approach to the exploration of \Wikipedia's category graph so that candidate taxonomies can be extracted from the graph. We were specifically interested in understanding \Wikipedia's classification of software languages. To this end, we developed a domain-specific exploration tool, \WikiTax, which supports level-by-level graph extraction, metrics-based graph visualization as well as transparent and revisable graph reduction. Such designated exploration support is missing in more generic tools for searching or exploring the category graph.

The described method of graph reduction is deliberately interactive and relies on domain knowledge for transparent exclusion decisions, as opposed to any means of automated ontology extraction / generation~\cite{SuchanekKW08,WuW08}. (Without such validation, there is little hope that the resulting taxonomy would be readily meaningful.) An important conceptual contribution is our proposal to document exclusion decisions with (comments for) exclusion types, thereby making reduction more systematic and transparent. This interactive approach can be contrasted with related work on taxonomy or ontology mining, where categories are classified and additional relationships are inferred automatically, e.g., by analyzing the structure of compound category names~\cite{NastaseS08}.

We contend that the described approach provides the initial core of a method for actually developing a taxonomy for software languages (and possibly other taxonomies) on the grounds of \Wikipedia. Collaborative work and further improved tool support are needed to actually arrive at a comprehensive taxonomy. We imagine that we need powerful refactoring operations on the category graph to facilitate taxonomy extraction and enforcement of consistent style. The exploration of the category graph could also be supported by additional forms of visualization, e.g., for understanding the overlap of categories. Also, we need to generally better understand (perhaps based on an automated analysis) the different classifier styles used by \Wikipedia.

%%%%%%%%%%%%%%%%%%%%%%%%%%%%%%%%%%%%%%%%%%%%%%%%%%%%%%%%%%%%

\begin{comment}
% Another problem with Wikipedia's use of categories
\WikipediaCategory{Ada programming language} is a subcategory of, for example, \WikipediaCategory{Concurrent programming languages}, while \WikipediaPage{Ada (programming language)} is not a member page of 
\WikipediaCategory{Concurrent programming languages}, but it is a member of various other ``... programming languages'' categories. For comparison consider \WikipediaCategory{Erlang programming language}, which is also a subcategory of, for example, \WikipediaCategory{Concurrent programming languages} and \WikipediaPage{Erlang (programming language)} is also a member page of \WikipediaCategory{Concurrent programming languages}.
\end{comment}

%%%%%%%%%%%%%%%%%%%%%%%%%%%%%%%%%%%%%%%%%%%%%%%%%%%%%%%%%%%%


%%%%%%%%%%%%%%%%%%%%%%%%%%%%%%%%%%%%%%%%%%%%%%%%%%%%%%%%%%%%

\bibliographystyle{splncs03}
\bibliography{paper}

%%%%%%%%%%%%%%%%%%%%%%%%%%%%%%%%%%%%%%%%%%%%%%%%%%%%%%%%%%%%

\end{document}

%%%%%%%%%%%%%%%%%%%%%%%%%%%%%%%%%%%%%%%%%%%%%%%%%%%%%%%%%%%%
