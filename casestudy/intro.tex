%%%%%%%%%%%%%%%%%%%%%%%%%%%%%%%%%%%%%%%%%%%%%%%%%%%%%%%%%%%%

\section{Introduction}
\label{S:intro}

Ever since 2008, the calls for papers for the \emph{Software Language Engineering} (SLE) conference\footnote{\url{http://planet-sl.org/}} have contained slightly different, more implicit or more explicit definitions of the term `software language'. Other community material contains yet other definition attempts; see, for example, the IEEE TSE special section on SLE in 2009~\cite{FavreGLW09}. At SLEBOK 2012 (i.e., an SLE 2012 satellite event dedicated to the the SL(E) body of knowledge), the attendees were also getting into the issue of what exactly a software language is and what classification may help in arriving at an accepted comprehensive definition.

Some classes of software languages are generally agreed upon. For instance, programming languages are definitely software languages; they are conceptually well understood and characterized in terms of programming language concepts; see, for example, textbooks on programming languages, programming paradigms, and programming language theory, e.g., \cite{Mosses92,Sebesta12}. Some classes of languages have been the target of scholarly work on language classification; see, for example, classifications of model transformation languages~\cite{CzarneckiH06}, business rule modeling languages~\cite{SkalnaG12}, visual languages~\cite{BottoniG04,MarriottM97,BurnettB94}, architecture description languages~\cite{MedvidovicT00}, and programming languages~\cite{BabenkoRY75,DoyleS87}.

In our work on the software chrestomathy `\oneohone'~\cite{FavreLV12}\footnote{\url{http://101companies.org/}}, we also aim at the classification of software languages, but we have failed to make a serious proposal so far. We are simply not confident regarding classification style, expected level of detail, and treatment of multiple dimensions of classification. In fact, such a SL(E) classification challenge is by no means limited to software languages; it also applies to \emph{software technologies} and \emph{software concepts}. Perhaps, we may need to lower expectations and accept the use of simpler tagging schemes (as used on StackOverflow, for example) as opposed to hierarchically organized, consistent and comprehensive taxonomies.

\Wikipedia{} contains substantial amounts of taxonomy-like (if not ontology-like) information---also for software languages, technologies, and concepts. For instance, there are hierarchically organized categories such as \WikipediaCategory{Computer languages}, \WikipediaCategory{Programming languages}, and \WikipediaCategory{Programming language classification}, but yet other roots for exploration may be reasonable. We suggest that the SL(E) classification challenge shall be informed by the exploration of \Wikipedia. 

In this paper, we describe support for such exploration based on the \WikiTax{} tool that was developed exactly for this use case. We also demonstrate exploration, without though any claim of having found a good candidate taxonomy for SL(E). Rather the expectation is that \WikiTax{} and the associated paradigm add more structure to the ongoing community effort of addressing the SL(E) classification challenge. The source code of \WikiTax, a comprehensive manual, and all data covered in this paper are available online.\footnote{\url{https://github.com/dmosen/wiki-analysis}}

\paragraph*{Road-map} \S\ref{S:tool} describes the \WikiTax{} tool. \S\ref{S:study} explores some \Wikipedia{} categories related to software languages. \S\ref{S:concl} concludes the paper.

%%%%%%%%%%%%%%%%%%%%%%%%%%%%%%%%%%%%%%%%%%%%%%%%%%%%%%%%%%%%
