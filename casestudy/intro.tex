%%%%%%%%%%%%%%%%%%%%%%%%%%%%%%%%%%%%%%%%%%%%%%%%%%%%%%%%%%%%

\section{Introduction}
\label{S:intro}

\vspace{-27\in}

Ever since 2008, the calls for papers for the \emph{Software Language Engineering} (SLE) conference\footnote{\url{http://planet-sl.org/}} have contained slightly different, more implicit or more explicit definitions of the term `software language'. Other community material contains yet other definition attempts; see, for example, the IEEE TSE special section on SLE in 2009~\cite{FavreGLW09}. At SLEBOK 2012 (i.e., an SLE 2012 satellite event dedicated to the the SL(E) body of knowledge), the attendees were also getting into the issue of what exactly a software language is.

A \emph{classification} of software languages is a useful (if not necessary) pillar of a definition of `software language'. Such classification is the topic of the present paper. One branch of software languages appears to be well understood. That is, \emph{programming languages} are obviously \emph{software languages} and they may be classified in terms of criteria and concepts as organized, for example, in textbooks on programming languages, programming paradigms, and programming language theory such as~\cite{Mosses92,Sebesta12}. There is also scholarly (dated) work on the classification of programming languages~\cite{BabenkoRY75,DoyleS87}. Actually quite a few sets of criteria or concepts exist for programming languages; there is no obvious contender; there is no comprehensive classification. Several classes of languages (other than programming languages) have been classified in scholarly work, e.g., model transformation languages~\cite{CzarneckiH06}, business rule modeling languages~\cite{SkalnaG12}, visual languages~\cite{BottoniG04,BurnettB94,MarriottM97}, and architecture description languages~\cite{MedvidovicT00}. The ultimate taxonomy of software languages should subsume and integrate existing, fragmented classifications in a transparent manner. The \emph{101companies} project\footnote{\url{http://101companies.org/}} hosts efforts targeted at such a taxonomy, but the results are of limited use and quality so far. 

In this paper, we try to inform the apparent classification challenge for software languages by means of exploring \Wikipedia. Obviously, \Wikipedia{} contains substantial amounts of taxonomy-like (if not ontology-like) information---also for software languages (without though embracing the actual term, at the time of writing). For instance, there are hierarchically organized categories such as \WikipediaCategory{Computer languages}, \WikipediaCategory{Programming languages}, and \WikipediaCategory{Programming language classification} that seem to apply; yet other categories may be relevant. Accordingly, we describe a method and a corresponding tool, \WikiTax, for exploring \Wikipedia's category graph. Exploration is supported in a manner such that a domain expert can reduce the category graph so that a classification emerges. The overall approach is not specific to software languages, but we apply it to software languages throughout the paper. 

\vspace{-27\in}

\paragraph*{\textbf{Contribution}} We do not claim to have converged on a good candidate taxonomy for software languages. Rather we contribute procedural, tool-supported elements of a method towards development of the ultimate taxonomy. The resulting tool, \WikiTax, is a rather simple graph exploration tool, which however includes a few domain-specific features not available in more generic functionality for searching and exploring \Wikipedia's category graph.

\vspace{-27\in}

\paragraph*{\textbf{Road-map}} \S\ref{S:approach} describes the overall exploration approach and sketches corresponding tool support as implemented by \WikiTax. \S\ref{S:study} explores \Wikipedia{} categories related to software languages. \S\ref{S:concl} concludes the paper. The source code of \WikiTax, a comprehensive manual, and all data covered in this paper are available online.\footnote{\url{https://github.com/dmosen/wiki-analysis}}

%%%%%%%%%%%%%%%%%%%%%%%%%%%%%%%%%%%%%%%%%%%%%%%%%%%%%%%%%%%%
